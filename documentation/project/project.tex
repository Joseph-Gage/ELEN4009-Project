\documentclass[12pt,onecolumn]{article}
\usepackage{graphicx}
\usepackage{hyperref}
\usepackage{url}
\usepackage{wrapfig}
\usepackage{tikz}
\usetikzlibrary{shapes}

\newcommand{\horrule}[1]{\rule{\linewidth}{#1}}

\title{\vspace{-4.2cm} \huge Tracking Interconnected Facebook Links - Project Report}

\author{ \horrule{1pt} \\ \textbf{ELEN4009 - Software Engineering} \\ \emph{School of Electrical and Information Engineering,} \\ \emph{University of the Witwatersrand,} \\ \emph{Johannesburg, South Africa} \\ \horrule{1pt} \\\\ \emph{Back-End Pair:} \\ Julian Zeegers (704582) \\ James Allingham (672732) \\ \\ \emph{Front-End Pair:} \\ Joseph Gage (751052)\\ Nathan Haag (873666) \\ \horrule{1pt}}

\addtolength{\oddsidemargin}{-1.5cm}
\addtolength{\evensidemargin}{-.0cm}
\addtolength{\textwidth}{3cm}


%%%%%%%%%%%%%%%%%%%%%%%%%%%%%%%%%%%%%%%%%%%%%%%%%%%%%%%%%%%%%%%%%%%%%%%%%%%%%%%
\begin{document}

\date{\vspace{-5ex}}
\maketitle
\pagestyle{plain}
\thispagestyle{empty}

\begin{abstract}
Your abstract goes here...
...
\end{abstract}

\newpage

\tableofcontents
\thispagestyle{empty}
\setcounter{page}{0}

\newpage

\section{Introduction}

	\subsection{Problem Statement} % James

	Facebook is one of the most popular social networks used today. Due to the fact that it has over 1.5 billion users, it produces an incredibly large amount of data \cite{fb}. This data could be used in a number of ways:

	\begin{itemize}
		\item Data scientists and analysts could make use of this data to learn about social trends.

		\item Individuals can use the data for personal analytics.

		\item Sociologists to use the data to make and test new hypotheses.

		\item Marketing and advertising agencies could create specialized and targeted adds depending on a set of user's likes and status updates.

	\end{itemize}    

	There are a large number of potential applications and the possibilities are only constrained by human creativity.

	The Tracking Interconnected Facebook Links project is intended to visualize the links and connections of a Facebook user with other users. Initially, these visualizations are focused on identifying the relationships of a Facebook user (the end user of this product) with the network of friends this Facebook user has. The visuals are also intended to further illustrate the relationship connections of the user's friends of friends and an overview of the user's friend network. Some personal analytics such as analyses of the user's Facebook post contents will also be implemented. There is a myriad of additional features that could be added to this tool and therefore the solution must be dynamic and flexible. For example degrees of separation analyses or more personal analytics. This document describes the software requirements that will ensure that the end product is of the highest quality, produced most efficiently and is created as close to the requirements as possible. At a later point the project could be expanded to include the large scale data analytics used by data scientists, sociologists and marketers. 

	The project team consists of four Software Engineering students, divided into two groups: one for the front-end and one for the back-end. This document serves as a description of the project. The software requirement specification will be laid out. The design and implementation of the front and back ends will be discussed. Finally, the Scrum planning and retrospective will be presented.

	\subsection{Project Objectives} % James

	In order to both quantify the success of the project as well as properly create tasks with correct priorities, the objective of the project must be laid out. They are as follows:

	\begin{itemize}

		\item Create a prototype system for visualizing a persons Facebook connections, social trends and personal analytics.

		\item Document the requirements for the system.

		\item Document the design of the system.

		\item Document the implementation and testing of the system.

		\item Document the group work aspects of the system including the SDLC.

	\end{itemize}

	\subsection{Stakeholders} % James

	\subsection{Abbreviations, Acronyms and Definitions} % All of us

	\begin{enumerate}
		\item ACID - Atomicity Consistency Isolation Durability. A model for reliable database systems.

		\item Apache. A popular web server.

		\item Client-Server. An architecture or model for distributed computing via the Internet or other network.

		\item CSS - Cascading Style Sheets. A style sheet language for describing the presentation of a website.

		\item Cypher. The query language for Neo4j graph databases.

		\item D3.js. A popular JavaScript library for visualization.

		\item DBMS - Database Management System. A database program.

		\item Django. An MTV web application framework, written in Python.

		\item Git. A program for software and document version control.

		\item GitHub. A website for hosting git repositories.

		\item GUI - Graphical User Interface. A visual tool that allows a user to interact with an application.

		\item HTML - Hyper Text Markup Language. A markup language for creating web pages.

		\item HTTP - Hyper Text Transfer Protocol. An Application level protocol that makes use of HTML from sending and receiving messages in a client-server architecture.

		\item JS - JavaScript. A scripting language often used in building interactive applications in the Web 2.0.

		\item JSON - JavaScript Object Notation. A data interchange notation used extensively in web applications.

		\item mod\_wsgi. The Django module for Apache Web Server.

		\item MTV - Model Template View. Another name for the popular MVC framework. This is the nomenclature used by the Django web framework.

		\item MVC - Model View Controller. A framework for building web applications based on a three layer abstraction of the data, the presentation and the interaction between them. 

		\item Neo4j. A graph database written in Java.

		\item Py2Neo. A Neo4j driver for the Python programming language.

		\item Python. A popular, powerful and flexible scripting language.

		\item Slack. An application for team communication.

		\item SDLC - Software Development Life Cycle. The process or approach the software development team adheres to.

		\item Trello. A web application for managing project tasks.

		\item UI - User Interface. A tool allowing a user to interact with an application.

		\item Web 2.0. The informal name for websites that emphasize user generated content such as Facebook.


	\end{enumerate}

\section{Software Requirement Specification} % James

\section{The Front-End}	

	\subsection{Design Document} % Joe

	\subsection{Implementation} % Nathan

\section{The Back-End}

	\subsection{Design Document} % Joe

	\subsection{Implementation} % Nathan

\section{Sprint Planning} % Julian

\section{Sprint Retrospective} % Julian

\begin{thebibliography}{1}
	\bibitem{Kinsey} H. van Vliet, \emph{Software Engineering: Principles and Practice} Wiley, 2007.
	
	\bibitem{twotieradvantage} N. Liyanage. \emph{Client/Server Architecture: Advantages and Disadvantages of the architectures}. 2013. \url{http://clientserverarch.blogspot.co.za/2013/03/advantages-and-disadvantages-of.html} Last accessed: 9 March 2016
	
	\bibitem{beginningsofteng} R. Stephens, Beginning Software Engineering, 1st ed. Indianapolis: John Wiley And Sons, Inc, 2015, pp. 94-95.
	
	\bibitem{Bootstrap}  \emph{Get Bootstrap - 3.3.6} \url{www.getbootstrap.com} Last accessed: 9 March 2016.
	
	\bibitem{D3}  \emph{D3.js - Data Driven Documents} \url{www.d3.com} Last accessed: 9 March 2016.
	
	\bibitem{django} Django Software Foundation. \emph{Django Overview}. \url{https://www.djangoproject.com/start/overview/}. 2016. Last accessed: 9 March 2016. 
	
	\bibitem{djangobook} Adrian Holovaty, Jacob Kaplan-Moss, et al. \emph{The Django Book}. \url{http://www.djangobook.com/en/2.0/index.html#}. Ch 3. Last accessed: 9 March 2016.
	
	\bibitem{djangoApache} Django Software Foundation. \emph{How to install Django}. \url{https://docs.djangoproject.com/en/1.9/topics/install/}. 2016. Last accessed: 9 March 2016.	
	
	\bibitem{apache} Apache Software Foundation. \emph{Apache - HTTP Server Project}. \url{https://httpd.apache.org/ABOUT_APACHE.html}. 2016. Last accessed: 9 March 2016.	
	
	\bibitem{graphdbs} Robinson I, Webber J, Eifrem E. \emph{Graph Databases} O'Reilly Media. ch 2. pp 21 - 22. June 2013.

	\bibitem{fb} Statista. \url {http://www.statista.com/statistics/264810/number-of-monthly-active-facebook-users-worldwide/}. Last accessed 18 February 2016. 

	\bibitem{neo4j} neo4j. \url{http://neo4j.com/}, Last accessed 18 February 2016.

	\bibitem{py2neo} Py2Neo. \url{http://py2neo.org/2.0/}, Last accessed 18 February 2016.

	\bibitem{slack} Slack. \url{https://slack.com/}, Last accessed 18 February 2016.

	\bibitem{trello} Trello. \url{https://trello.com/}, Last accessed 18 February 2016.

	\bibitem{wsgi} mod\_wsgi. \url{https://modwsgi.readthedocs.org/en/develop/}, Last accessed 7 April 2016.


\end{thebibliography}

\end{document}