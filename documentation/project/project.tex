\documentclass[12pt,onecolumn]{article}
\usepackage{graphicx}
\usepackage{hyperref}
\usepackage{url}
\usepackage{wrapfig}
\usepackage{tikz}
\usetikzlibrary{shapes}

\newcommand{\horrule}[1]{\rule{\linewidth}{#1}}

\title{\vspace{-4.2cm} \huge Project Report for Tracking Interconnected Facebook Links }

\author{ \horrule{1pt} \\ \textbf{ELEN4009 - Software Engineering} \\ \emph{University of the Witwatersrand} \\ \horrule{1pt} \\\\ \emph{Back-End Pair:} \\ Julian Zeegers (704582) \\ James Allingham (672732) \\ \\ \emph{Front-End Pair:} \\ Joseph Gage (751052)\\ Nathan Haag (873666) \\ \horrule{1pt}}

\date{December 2004}




\addtolength{\oddsidemargin}{-1.5cm}
\addtolength{\evensidemargin}{-.0cm}
\addtolength{\textwidth}{3cm}


%%%%%%%%%%%%%%%%%%%%%%%%%%%%%%%%%%%%%%%%%%%%%%%%%%%%%%%%%%%%%%%%%%%%%%%%%%%%%%%
\begin{document}

\date{\vspace{-5ex}}
\maketitle
\pagestyle{plain}
\thispagestyle{empty}

\begin{abstract}
Your abstract goes here...
...
\end{abstract}

\newpage

\tableofcontents
\thispagestyle{empty}
\setcounter{page}{0}

\newpage

\section{Introduction}

	\subsection{Problem Statement} % James

	\subsection{Project Objectives} % James

	\subsection{Stakeholders} % James

	\subsection{Abbreviations, Acronyms and Definitions} % All of us

	\begin{enumerate}
		\item ACID - Atomicity Consistency Isolation Durability. A model for reliable database systems.

		\item Apache. A popular web server.

		\item Client-Server. An architecture or model for distributed computing via the Internet or other network.

		\item CSS - Cascading Style Sheets. A style sheet language for describing the presentation of a website.

		\item Cypher. The query language for Neo4j graph databases.

		\item D3.js. A popular JavaScript library for visualization.

		\item DBMS - Database Management System. A database program.

		\item Django. An MTV web application framework, written in Python.

		\item Git. A program for software and document version control.

		\item GitHub. A website for hosting git repositories.

		\item GUI - Graphical User Interface. A visual tool that allows a user to interact with an application.

		\item HTML - Hyper Text Markup Language. A markup language for creating web pages.

		\item HTTP - Hyper Text Transfer Protocol. An Application level protocol that makes use of HTML from sending and receiving messages in a client-server architecture.

		\item JS - JavaScript. A scripting language often used in building interactive applications in the Web 2.0.

		\item JSON - JavaScript Object Notation. A data interchange notation used extensively in web applications.

		\item mod\_wsgi. The Django module for Apache Web Server.

		\item MTV - Model Template View. Another name for the popular MVC framework. This is the nomenclature used by the Django web framework.

		\item MVC - Model View Controller. A framework for building web applications based on a three layer abstraction of the data, the presentation and the interaction between them. 

		\item Neo4j. A graph database written in Java.

		\item Py2Neo. A Neo4j driver for the Python programming language.

		\item Python. A popular, powerful and flexible scripting language.

		\item Slack. An application for team communication.

		\item Trello. A web application for managing project tasks.

		\item UI - User Interface. A tool allowing a user to interact with an application.

		\item Web 2.0. The informal name for websites that emphasize user generated content such as Facebook.


	\end{enumerate}

\section{Software Requirement Specification} % James

\section{The Front-End}

	

	\subsection{Design Document} % Joe

	\subsection{Implementation} % Nathan

\section{The Back-End}

	\subsection{Design Document} % Joe

	\subsection{Implementation} % Nathan

\section{Sprint Planning} % Julian

\section{Sprint Retrospective} % Julian

\begin{thebibliography}{1}
	\bibitem{Kinsey} H. van Vliet, \emph{Software Engineering: Principles and Practice} Wiley, 2007.
	
	\bibitem{twotieradvantage} N. Liyanage. \emph{Client/Server Architecture: Advantages and Disadvantages of the architectures}. 2013. \url{http://clientserverarch.blogspot.co.za/2013/03/advantages-and-disadvantages-of.html} Last accessed: 9 March 2016
	
	\bibitem{beginningsofteng} R. Stephens, Beginning Software Engineering, 1st ed. Indianapolis: John Wiley And Sons, Inc, 2015, pp. 94-95.
	
	\bibitem{Bootstrap}  \emph{Get Bootstrap - 3.3.6} \url{www.getbootstrap.com} Last accessed: 9 March 2016.
	
	\bibitem{D3}  \emph{D3.js - Data Driven Documents} \url{www.d3.com} Last accessed: 9 March 2016.
	
	\bibitem{django} Django Software Foundation. \emph{Django Overview}. \url{https://www.djangoproject.com/start/overview/}. 2016. Last accessed: 9 March 2016. 
	
	\bibitem{djangobook} Adrian Holovaty, Jacob Kaplan-Moss, et al. \emph{The Django Book}. \url{http://www.djangobook.com/en/2.0/index.html#}. Ch 3. Last accessed: 9 March 2016.
	
	\bibitem{djangoApache} Django Software Foundation. \emph{How to install Django}. \url{https://docs.djangoproject.com/en/1.9/topics/install/}. 2016. Last accessed: 9 March 2016.	
	
	\bibitem{apache} Apache Software Foundation. \emph{Apache - HTTP Server Project}. \url{https://httpd.apache.org/ABOUT_APACHE.html}. 2016. Last accessed: 9 March 2016.	
	
	\bibitem{graphdbs} Robinson I, Webber J, Eifrem E. \emph{Graph Databases} O'Reilly Media. ch 2. pp 21 - 22. June 2013.

	\bibitem{fb} Statista. \url {http://www.statista.com/statistics/264810/number-of-monthly-active-facebook-users-worldwide/}. Last accessed 18 February 2016. 

	\bibitem{neo4j} neo4j. \url{http://neo4j.com/}, Last accessed 18 February 2016.

	\bibitem{py2neo} Py2Neo. \url{http://py2neo.org/2.0/}, Last accessed 18 February 2016.

	\bibitem{slack} Slack. \url{https://slack.com/}, Last accessed 18 February 2016.

	\bibitem{trello} Trello. \url{https://trello.com/}, Last accessed 18 February 2016.

	\bibitem{wsgi} mod\_wsgi. \url{https://modwsgi.readthedocs.org/en/develop/}, Last accessed 7 April 2016.


\end{thebibliography}

\end{document}